\documentclass{article}
\usepackage{amsmath}
\usepackage{cite}
\usepackage{tikz}
\usepackage{bm}
\usepackage{tikz,tkz-tab}
\usepackage{amsfonts}%
\usepackage{amssymb}%
\usepackage{hyperref}
\usepackage{mathtools}
%\usepackage{subcaption}
\usepackage{color}
\usepackage{setspace}
\usepackage{empheq}
\usepackage{bbm, dsfont}
\usepackage{dsfont}
\usepackage{mathtools}
\usepackage{geometry}
\usepackage{enumitem} 
\usepackage[bottom]{footmisc}
\usetikzlibrary{arrows}
\usepackage{lscape}
\usepackage{tcolorbox}
\usepackage{caption}
%\usepackage{graphicx}
\usepackage{subfig}
\usetikzlibrary{shapes,snakes}
\renewcommand{\labelitemi}{$\bullet$}
\renewcommand{\labelitemii}{$\diamond$}


\newtheorem{definition}{Definition}

\usetikzlibrary{positioning}
\tikzset{main node/.style={circle,draw,minimum size=0.5cm,inner sep=0pt},
            }

%-------------------------------------------
\newtheorem{example}{Example}
\newtheorem{theorem}{Theorem}
\newtheorem{acknowledgement}[theorem]{Acknowledgement}
\newtheorem{algorithm}[theorem]{Algorithm}
\newtheorem{axiom}[theorem]{Axiom}
\newtheorem{case}[theorem]{Case}
\newtheorem{claim}{Claim}
\newtheorem{conclusion}[theorem]{Conclusion}
\newtheorem{condition}[theorem]{Condition}
\newtheorem{conjecture}[theorem]{Conjecture}
\newtheorem{corollary}{Corollary}
\newtheorem{criterion}[theorem]{Criterion}
\newtheorem{assumption}{Assumption}
\newtheorem{exercise}[theorem]{Exercise}
\newtheorem{lemma}{Lemma}
\newtheorem{observation}{Observation}
\newtheorem{notation}[theorem]{Notation}
\newtheorem{problem}[theorem]{Problem}
\newtheorem{proposition}{Proposition}
\newtheorem{remark}[theorem]{Remark}
\newtheorem{solution}[theorem]{Solution}
\newtheorem{summary}[theorem]{Summary}
\newenvironment{proof}[1][Proof]{\textbf{#1.} }{\ \rule{0.5em}{0.5em}}

\begin{document}

\title{ TwiClim}
\date{}
\maketitle
\onehalfspace
\section*{Literature}
\begin{itemize}
	\item Social Problems 
		\begin{itemize} 	
			\item Social Problems, {\it the 2019 SSSP Presidential Address is a {\color{purple}call to focus on what is the largest social problem: climate change}.} Social Problems, 2020, 67). 
			\item Constitution of social problems, Daniel Cefai
			
			
			\item 2018, "The Digital Activism Gap: How Class and Costs Shape Online Collective Action", Social Problems, vol. 65, n° 1, p. 51-74
			\item Social problem : formulated differently by different groups (scientific consensus and facts, policy, climate justice)
		\end{itemize}
		\item Sociologie de l'acteur r\'{e}seau: controverse? (Michel Callon, Walsh et Renaud 2010)
				\begin{itemize}
					\item	Moment 1: {\color{purple}perception de la n\'{e}cessit\'{e} du changement} (expression par un acteur qui voit les opportunités ancrées dans le changement)
					\item Moment 2: Identification des actants et probl\'{e}matisation 
					\item Moment 3: Int\'{e}ressement et alliances 
					\item Moment 4: Distribution des r\^{o}les et enr\^{o}lement: roles distribu\'{e}s et accept\'{e}s de fa\c{c}on informelle
					\item {\bf Moment 5: mobilisation des portes-paroles}
					\item Digital Activism: Twitter comme actant ? vecteur de mobilisation dans l'espace public (attention Twitter n'est pas un espace public). Diffusion of misinformation and misperception: problem of this actant? 
				
				\end{itemize}
			\item Discourses in interaction {\color{gray}(Sanna-Kaisa Tanskanen carries out research in the fields of applied linguistics, discourse studies and (internet) pragmatics.)}
				\begin{itemize}
					\item Contexts in context: Micro meets macro ({\bf \color{purple}Anita Fetzer})
						\begin{itemize}
							\item Discourse analysis where discourse is investigated as a theoretical construct (e.g., Brandom 1994, Foucault 1997, Linell 1998 and this volume)
							\item Context- and discourse-based research extends its frame of reference by looking beyond the actual object of investigation thus {\color{purple}integrating relevant background information} regarding {\color{purple}production} and {\color{purple}reception} on the one hand, and {\color{purple}order and power structure} on the other.
							\item Discourse analysis tends to accommodate both diachronic (cf. Hiltunen, this volume) and synchronic perspectives stressing the fact that discourses are historically grown. 
							\item Discourse is frequently considered to be contained in context, while context is seen as an unbounded entity embedding discourse. Alternatively, context is seen as being presupposed or indexically contained in discourse, and it is the bounded entity of discourse which instantiates the reconstruction of context. 
							\item Context and discourse are frequently examined from a parts-whole perspective. On the one hand, context and discourse are conceived as wholes and are analyzed accordingly. On the other hand, context and discourse are conceived as parts-whole configurations, and it is the constitutive parts of that configuration which are identified, described and analyzed. As a consequence of the {\color{purple}dual perspective, an investigation of context and discourse requires the accommodation of macro- and micro-oriented viewpoints}. While the former employs a {\bf top-down perspective accounting for the ‘object as a whole’}, that is institutional context or discourse genre, the latter employs a {\bf bottom-up perspective accounting for the ‘object’s constitutive parts’}, that is the immediate context or the immediate associations and collocations of an object of discourse.
						\end{itemize}
				\end{itemize}
		\item Herméneutique: Néanmoins, certains auteurs de la deuxième moitié du XXe siècle, comme Paul Feyerabend, soutiennent que {\color{purple} \bf le discours scientifique est lui aussi une interprétation du monde} et que {\color{purple}son mode de production ne diffère pas de celui des autres discours}, littéraires, mythologiques. Paul Feyerabend, Contre la méthode, Seuil, Points Sciences, 1975. On trouve cette idée avant Feyerabend chez Nietzsche, Par-delà bien et mal, "Des préjugés des philosophes"
\end{itemize}
\newpage
\section{Tentative: {\color{red} Elite discourses in interaction}}
Is Twitter an actant in the climate discourse? How is the problem formulated? Investigate, initial ``seed" groups and expand to their fan base. Text as main object, use discourse analysis... 

			\begin{itemize}
				\item[$\star$] {\color{red} \bf Production.} Characterize different discourses? 
				
					\begin{itemize}
						\item different sources? how to differentiate the sources 
						\begin{itemize}
							\item	co-citation network ? 
							\item ratings of sources 
							\item Political bias of sources
							\item Political bias of actors 
							
						\end{itemize}
						\item	step 0: topics? step 1: stance from replies
						\begin{itemize}
							\item sub-discourse?
						\end{itemize}
						\item Semantic network
					\end{itemize}
					
				\item[$\star$]  {\color{red} \bf Reception.} Influence these discourses?
					\begin{itemize}
						\item reach? engagement?  
						\item following network
					\end{itemize}
				\item[$\star$] {\color{red} \bf Order and power structure.} Discourses in interaction?
					\begin{itemize}
					\item reply network, mention network?
					\item (power) Different occupation? 
							\begin{itemize}
								\item	description bio 
								\item level of education Phd, etc.
							\end{itemize}
					\end{itemize}

			\end{itemize}


\section{Introduction}
\section{Methods}
\section{Results}
\end{document}